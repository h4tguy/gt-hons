Let $t$ be a positive integer and $Q_t$ the hypercube with $2^t$ vertices. These
vertices can be represented by $t$-bit strings whereby two vertices are adjacent
iff their representation as strings differ in exactly one place. Furthermore,
$Q_t$, by definition, is connected.

Choose $v \in V\left(G\right)$ and let $V_1$ denote the set of vertices which differ from $v$
in their string representation in an odd number of positions, and let $V_2$ denote
the vertices which differ from $v$ in an even number of places and $v$ itself i.e. the
remaining vertices. Clearly these sets are well-defined. We shall show that no
vertex in $V_1$ is adjacent to any other vertex in $V_1$, and similarly no vertex in $V_2$
is adjacent to any other vertex in $V_2$.

Clearly $v$ is not adjacent to any other vertex in $V_2$. Let $u$ be one such vertex.
Assume there exists $u' \in V_2$ such that $u$ and $u'$ are adjacent. Then their
string representations differ in exactly one place, which implies that $u'$'s string
representation differs from that of $v$ in an odd number of places, so $u'$ cannot
be in $V_2$.

Similarly for any two vertices $x,y \in V_1$: If $x$ and $y$ are adjacent, then they
differ in their string representations from each other in exactly one place, which
implies that one of them differs from $v$'s string representation in an even number
of places, and so both cannot be in $V_1$. We have thus proved the claim.
