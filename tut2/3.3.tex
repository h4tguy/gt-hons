\paragraph{Definition} A clique of a graph $G$ is a connected subgraph of $G$ which is
isomorphic to a complete graph. For simplicity, we shall denote cliques by the
complete graph $K_t$ to which the clique is isomorphic where $t$ is a positive integer.
\paragraph{Lemma} Assume a bipartite graph has a clique $K_t$. Then $t \leq 2$.
\paragraph{Proof of Lemma} By contradiction: we prove that if a graph has a clique
$K_3$ then it is not bipartite. Further, for $n \geq 3$, $K_n$ has a clique $K_3$, which can
be seen by considering the subgraph generated by any three vertices in $K_t$ and
realising that this subgraph is indeed $K_3$.
\paragraph{} Let $G$ be bipartite and assume it has a clique $K_3$. Label the vertices in $K_3$
as $u,v,w$. A bipartitie graph is a graph which has two disjoint subsets $V_1$ and
$V_2$ contained in $V\left(G\right)$ such that $V_1 \cup V_2 = V\left(G\right)$ and $G\left[V_1\right]$ and $G\left[V_2\right]$ are
both discrete, or empty. Choose such $V_1$ and $V_2$ and w.l.o.g. let $u \in V_1$. Then
$v,w \in V_2$ which implies that $G\left[V_2\right]$ has the edge $vw$ and so is not empty. This
contradiction proves the lemma. Q.E.D.
\paragraph{Answer}For a bipartite graph $G$ with order greater than 5, we have that at
least one of $V_1$, $V_2$ is of order greater than or equal to 3, which implies that $\bar{G}$
has a $K_3$ clique as complementation joins every non-adjacent vertex in $G$ i.e.
$\bar{G}\left[V_1\right]$ is isomorphic to $K_{n\left(V_1\right)}$ and similarly for $\bar{G}\left[V_2\right]$.
