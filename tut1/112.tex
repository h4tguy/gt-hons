\begin{enumerate}[(a)]
\item The straight line has only one symmetry on it.
Consider $P_n$, and label its vertices $1,2,\dots, n$, with edge $i,i+1$
for $1 \le i < n$. Now note $1$ and $n$ are the only degree $1$ vertices, so
if $f$ is an automorphism then $f(1)$ is $1$ or $n$. Also, distance
is preserved through automorphism, and that each vertex is uniquely defined by
its distance to $1$, so fixing $f(1)$ fixes the entire automorphism. So there
are at most two automorphisms, the identity and the automorphism taking $i$ to
$n+1-i$. Both of these are clearly automorphisms. When $n=1$, they are not
distinct and there is $1$ automorphism on $P_n$. When $n>1$ they are distinct,
and there are two automorphisms.
\item The cycle $C_n$ has as its automorphism group the dihedral group 
$D_n$, as pointed out in class. $\text{Total} = 2n$
\item Label the vertices $1$ to $n$. Let $\pi_i$ be the permutation
that switches $1$ and $i$ and fixes everything else. Then, since every pair of
vertices is connected, this permutation is an automorphism. Since this is true
for all $i$, the $\pi_i$ generate $S_n$ and the automorphism group is a subgroup
of $S_n$, then the automorphism group is $S_n$
and $\text{Total}=n!$
\item As above, we can generate the automorphism group $S_n$ from 
$2$-cycles. $\text{Total} = n!$
\end{enumerate}
