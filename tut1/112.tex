\paragraph{a} The straight line has only one symmetry on it. Similarly for the graph: to see this, denote $P_n$ by the vertex sequence $v_1,v_2,...,v_n$. Then vertices $v_1$ and $v_n$ are the only degree 1 vertices. An automorphism can switch these, or not switch these. If we switch them, then $v_2$ and $v_{n-1}$ must also switch, and $v_3$ and $v_{n-2}$ and so forth. Otherwise, if we do not switch them, then we cannot switch $v_2$ and $v_i$ for any degree 2 $v_i$, as:
\subparagraph{case 1} if $v_i$ is adjacent to $v_n$ then $v_n$ would have to switch to something and can only switch to $v_1$.
\subparagraph{case 2} else $v_i$ is only adjacent to degree 2 vertices, in which case its neighbourhood is not isomorphic to the neighbourhood of $v_2$, so they cannot be switched.
And so $v_2$ cannot be switched with anything if $v_1$ is not switched, and so neither can $v_3$ and, by induction, no $v_i$. Thus the only automorphisms are the one generated by switching $v_1$ and $v_n$ and the identity automorphism. $Total = 2$
\paragraph{b} The cycle $C_n$ has as its automorphism group the dihedral group $D_n$, as pointed out in class. $Total = 2n$
\paragraph{c} Any two vertices can be switched, so the number of isomorphisms is the order of $S_n$. $Total = n!$
\paragraph{d} As above, any two vertices can again be switched. $Total = n!$
