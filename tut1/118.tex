\paragraph{a}
\subparagraph{order 1}
The trivial graph of order 1 and size 0 is self-complementary.
\subparagraph{order 2}
A graph of order 2 cannot be self-complementary.
\subparagraph{order 3}
Assume that a graph of order 3 is self-complementary. Then, if it has a vertex of degree 0, its complement must have a vertex of degree 0 as well (for this vertex to be mapped to). But that means the original graph has a vertex of degree 2 (whose degree in the complementary graph has degree 0) and so the graph is connected and there is no vertex of degree 0. A graph of order 3 has to have a vertex of either degree 0 or degree 2.
\subparagraph{order 4}
Assume a graph of degree 4 has is self-complementary. If it has a vertex of degree 0, then, as above, it cannot be self-complementary. Thus it also cannot have a vertex of degree 3. So it can only have vertices of degrees 2 and 1. For each vertex with degree 2, its degree in the complementary graph is 1. So for each vertex of degree 2 we need a vertex of degree 1 in the graph. So the only graph to consider is $P_4$. And it is self-complementary.
\subparagraph{order 5}
A graph of order 5 can be self-complementary, if each vertex has degree 2 (i.e. the cycle $C_5$). There is also another graph of degree 5.
