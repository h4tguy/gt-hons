Suppose $G$ has an Eulerian circuit $x_1x_2\dots x_kx_1$. Suppose also
that the vertex $y$ appears $l$ times in the circuit, as 
$x_{i_1},\dots,x_{i_l}$. Then, since every edge appears in the circuit
exactly once, and the edges in the circuit incident with $l$ are precisely
$$x_{i_1-1}x_{i_1},x_{i_1}x_{i_1+1},x_{i_2-1}x_{i_2},x_{i_2}x_{i_2+1},\dots,x_{i_l-1}x_{i_l},x_{i_l}x_{i_{l+1}}$$, of which there are $2l$, the degree
of $y$ is $2l$, which is even. Since the choice of $y$ was arbitrary, the
degree of every vertex in $G$ is even.

We now show that every connected
graph where all vertices are of even degree has
an Eulerian circuit. Suppose for contradiction this was not the case, and
let $G$ be a connected graph of minimal
size in which every vertex has even degree
and the graph has no Eulerian circuit. Choose an arbitrary vertex $v$ and
construct a trail $T$ in the following way: start at $v$, and while there is
an edge incident to the current vertex that has not been used,
follow that edge. 

Suppose that
the algorithm terminates at vertex $v'$. Suppose for contradiction
$v' \neq v$. Then suppose $v'$ appears $k$ times in $T$ preceding the
last occurence. At each of these internal occurences, two distinct edges
incident with $v'$ are traversed, and another edge incident with $v'$ is
traversed before the final occurence. Also there are no edges incident with
$v'$ not traversed, since the algorithm terminated at $v'$. Therefore the
degree of $v'$ is $2k+1$, a contradiction.

Hence $v'=v$ and $T$ is a circuit. Consider $G-E(T)$. This consists of
a number of isolated vertices and some nontrivial components 
$P_1,P_2,\dots,P_n$. Also, note that, for any $i$, $1 \le i \le n$,
if $x \in V(P_i)$ appears $l$ times in $T$, then $2l$ edges adjacent to $x$
are included in $T$ and so $\text{deg}_{G-E(T)} x$ is even. Also, $P_i$
has strictly fewer edges than $G$ so is Eulerian. 

Finally, suppose for
contradiction that no vertex of $P_i$ is in $T$. Then, let $u$ be a vertex
in $P_i$, and let $u=u_1,u_2,\dots,u_p=v$ be a $u-v$ path in $G$ (which
exists since $G$ is connected). Then
let $j_0$ be the maximum $j$ such that $u_j \in P_i$. Note $v\in T$ so
$v \notin P_i$, and $j_0<p$. Then $u_{j_0+1} \notin P_i$, so
$u_{j_0}u_{j_0+1} \notin P_i$ and hence $u_{j_0}u_{j_0+1} \in T$ and 
finally $u_{j_0} \in T$, a contradiction.

Let $T$ be $x_1,x_2,\dots,x_{c_1},\dots,x_{c_2},\dots,x_{c_n},\dots,
x_q=x_1$, with
wlog $x_{c_i} \in P_i$, and let $Q_i$ be an Eulerian circuit for $P_i$ 
starting and ending at $x_{c_i}$. Then
$$x_1,x_2,\dots,x_{c_1-1},Q_1,x_{c_1+1},\dots,x_{c_n-1},Q_n,
x_{c_n+1},\dots,x_q=x_1$$
is an Eulerian circuit for $G$ since it traverse every edge in 
$E(T) \cup \left( \bigcup_{i=1}^n E(P_i)\right)$ exactly once,
contradicting that
$G$ was not Eulerian.

It follows that any connected graph with every vertex of even degree must
be Eulerian.
