We label the vertices of $G = P_2 \times P_3$ as {$a,b,c,d,e,f$} 
with $a \sim b \sim c$ and $d \sim e \sim f$ where $d(c) = d(e) = 3$
i.e. $c$ and $e$ are the central vertices.
\paragraph*{a)} The automorphisms of $G$ are 
\begin{itemize}
\item $\iota = (a)$\\
\item $\alpha = (ad)(be)(cf)$\\
\item $\beta = (ac)(df)$\\
\item $\gamma = (af)(dc)(be)$
\end{itemize}

\paragraph*{b)} $a^{\Gamma} = \{ a,c,d,f \} = c^{\Gamma} = d^{\Gamma} = f^{\Gamma}$ and\\
$b^{\Gamma} = \{ e,f \} = e^{\Gamma}$.

\paragraph*{c)} $\Gamma_a = \{ \iota \} = \Gamma_c = \Gamma_d = \Gamma_f$ and\\
$\Gamma_b = \{ \beta \}$.

\paragraph*{d)} $\text{fix}(\iota) = V(G)$,\\
$\text{fix}(\alpha) = \emptyset = \text{fix}(\gamma)$, and\\
$\text{fix}(\beta) = \{c,e \}$. \\
Thus we obtain from the CFB Theorem that the number
of orbits should be $\frac{6 + 2 + 0 + 0}{4} = 2$.

\paragraph*{e)} $G$ has 6 vertices, so there are $2^6 = 64$ ways 
to colour the vertices with 2 colours. 
However, some of these graphs may be isomorphic: 
for each distinct cycle making up a permutation in the isomorphism group, 
we fix a colour. Then, using CFB Theorem, we divide by the order of the automorphism group. 
That is:
$\text{The number of distinct 2-colourings of $G$} = \frac{2^6}{4} + \frac{2^3}{4} + \frac{2^3}{4} + \frac{2^5}{4} = 16 + 2 +8 + 2 = 28$.
