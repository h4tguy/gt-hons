\begin{enumerate}[(a)]
	\item The Petersen graph is cubic, so $\kappa(G)=\lambda(G)$
		(see 5.12). We show that $\lambda(G)=3$. Firstly, 
		$\lambda(G) \le \delta(G)=3$ by Whitney (\cite{notes}, Theorem 24),
		and so we need to show that removing any two edges from $G$ leaves it
		connected. The Petersen graph consists of $2$ $C_5$s (let's call them $A$ and $B$)
		joined
		by $5$ edges, with every vertex in each of $A$, $B$ adjacent to some
		vertex in the other. Note that $C_5$ is two-edge-connected, so if we
		remove an edge from either $A$ or $B$ it remains connected as a subgraph.
		
		If at least one edge between $A$ and $B$ is removed, then at most $1$
		edge internal to each of $A$ and $B$ is removed, and each is still
		internally connected. Since at least $3$ edges remain connecting $A$ and
		$B$, the whole graph remains connected.

		If one edge is removed from each of $A$ and $B$, then both remain
		internally connected, and the graph is still connected since there
		are $5$ edges connecting $A$ and $B$.

		If two edges are removed from wlog $A$, then $B$ is still
		connected, and each vertex in $A$ is adjacent to a vertex in $B$,
		so the whole graph is still connected.

		In any case, removing two edges does not disconnect $G$, so
		$\lambda(G) \ge 3$ and hence $\kappa(G)=\lambda(G)=3$.

\item We prove by induction that $\kappa(Q_t)=\lambda(Q_t)=t$. For
	$t=1$ this is obvious, for $Q_1=K_2$ and so $\kappa(Q_1)=\lambda(Q_1)=1$.
	Suppose it is true for $t\le s-1$. We show it true for $t=s$.

	Firstly, note that $\kappa(Q_s)\le\lambda(Q_s)\le \delta(Q_s)=s$ (\cite{notes}, Theorem 24)
	and so it is sufficient to prove $\kappa(Q_s)\ge s$. Let $X$ be an $s-1$ vertex
	set and let $u,v$ be two arbitrary vertices not in $X$. We show that $u$ and $v$
	are connected in $Q_s-X$. We index the vertices of $Q_s$ with coordinates in
	$\mathbb{Z}_2^s$, as is standard. We have two cases:
	\begin{itemize}
		\item $u,v$ share at least one coordinate, say the $k$-th. Then the induced
			subgraph on the vertices sharing the $k$-th coordinate with $u$, call it $P$, 
			is isomorphic to $Q_{s-1}$.
			If there are at most $s-2$ vertices from $X$ in $P$, then $P-X$ is connected
			by the induction hypothesis, and so $u$ and $v$ are connected.
			Otherwise, $X \subset V(P)$. Then let $u'$ and $v'$ have the same coordinates
			as $u,v$ respectively, except with the $k$-th switched. Then 
			$u'u,v'v \in E(Q_s)$ and $u'$ and $v'$ are trivially connected (since both are
			in $Q_s-P \cong Q_{s-1}$), so $u$ and $v$ are connected.
		\item $u,v$ do not have any coordinate in common. Then $u$ has at least one neighbour
			in $Q_s-X$ (since $\text{deg}_{Q_s} u = s$). Moreover, that neighbour has at
			least one coordinate different from $u$, hence at least one coordinate in common
			with $v$. So $u$ is connected to $v$ via this neighbour, by the first case.
	\end{itemize}
	So, by induction $\kappa(Q_t) \ge t$ for all positive integers $t$, and hence
	$\kappa(Q_t)=\lambda(Q_t)=t$.

\item We show $\kappa(G)=\lambda(G)=4$
	By Whitney, $\kappa(G) \le \lambda(G) \le \delta(G)=4$. It is sufficient to
	prove that $\kappa(G)=4$ or that any pair of vertices has $4$ internally
	disjoint paths connecting them.

	Consider any $2$ vertices $u,v$. We repeatedly use that each of the $6$
	vertices has degree $4$, so is not adjacent to exactly one other vertex.
	We have two cases:
	\begin{itemize}
		\item If $u,v$ are not adjacent, then they are each adjacent to the other
			$4$ vertices $a,b,c,d$, and $uav,ubv,ucv,udv$ are $4$ internally disjoint
			$u-v$ paths.

		\item If $u,v$ are adjacent, then they are also both adjacent to two
			vertices $x,y$ and there are two vertices $a$ and $b$ such that $a$ is
			adjacent to $u$ and not $v$, $b$ is adjacent to $v$ and not $u$ and
			$a$ is adjacent to $b$. Then $uv$, $uxv$, $uyv$ and $uabv$ are $4$
			internally disjoint $u-v$ paths.
	\end{itemize}
	In either case $p(u,v)=4$, and since the choice of $u,v$ was arbitrary,
	$\kappa(G) \ge 4$. Hence $\kappa(G)=\lambda(G)=4$.
\end{enumerate}
