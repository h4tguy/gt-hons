For a graph $G$, define 
$$M(G) = \text{max}\left\{\delta(H): \text{$H$ is an induced subgraph of $G$} \right\}.$$

We proceed by induction on the order $n$ of $G$. Clearly when $n=1$ the
result holds ($K_1$ is $1$-colourable). Suppose the result is true for
$n=k$. We show it is true for $n=k+1$.

Let $G$ be a graph of order $k+1$ and let $v$ be a vertex of minimal degree in
$G$. Firstly, $M(G-v)\le M(G)$ since the induced subgraphs of $G-v$ are a subset of
the induced subgraphs of $G$. So, in particular, it is possible to colour $G-v$ with
at most $1+M(G)$ colours, by the induction hypothesis.  However, clearly
$\text{deg } v = \delta(G) \le M(G)$ so $v$ has at most $M(G)$ neighbours and,
in particular, there is a colour (from the colouring of $G-v$) that it is not
adjacent to. If we use this colour for $v$, we have a proper colouring for $G$ using
at most $1+M(G)$ colours, so $\chi(G) \le 1+M(G)$.

Therefore, by induction, $\chi(G) \le 1+M(G)$ for all graphs $G$.
