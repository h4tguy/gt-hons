Let $G$ be a cubic graph. 

By Whitney's theorem we know that $\kappa(G) \le \lambda(G) \le \delta(G) = 3 $.

Assume that $S$ is a vertex-cut with $|S| = \kappa(G)$. Then
$G-S$ is disconnected into two components $X, Y$. 
We will now show that we can find a minimal edge-cut $T$ such that $|T| = |S|$.

If $\kappa(G) = 3$. $\kappa(G) = 3 \le \lambda(G) \le \delta(G) = 3$
then $\lambda(G) = 3 = \kappa(G)$ as required.

For the cases where $\kappa(G) \in \{1, 2\}$ we continue as follows:

If none of the vertices in $S$ are adjacent to each other we can do the following:
For every $x \in S$, $x$ is connected to either $X$ or $Y$ by at most one edge $e$. 
Add this edge to our edge-cut.

If $v_1, v_2 \in S$ are adjacent then we have two cases:
\begin{enumerate}
    \item $v_1, v_2$ together have either two edges
incident to one component and two to the other. In this case we can choose any two of the edges
incident to the same component.
    \item $v_1$ is adjacent to a vertex in one component, a vertex in the 
        other component and $v_2$. $v_2$ is adjacent to to $v_1$ and 
        two vertices in the other component. In this case we have a contradiction
    since $v_1$ is a cut-vertex and $\kappa(G) = 1$. This case is impossible. 
\end{enumerate}

$v_1$ has one edge incident to a vertex 
in $X$ and one edge incident to a vertex in $Y$. Similarly with $v_2$. Choose 
the edges of $v_1, v_2$ incident with $X$ to be in our edge-cut. 

For every vertex in our vertex cut we have inserted one edge into our edge
cut so $|S| = |T|$ as required. 

Let $x\in X, y \in Y$, a $x-y$ path would have to go through one of the
vertices in the vertex-cut $S$. For any such vertex $s \in S$ it is the
case that the edge from the component $X$ to $s$ has been removed or the
vertex from $s$ to the component $Y$ has been removed so no such path
exists, $X, Y$ is disconnected and $T$ is a vertex cut. 

$\kappa(G) = \lambda(G)$
