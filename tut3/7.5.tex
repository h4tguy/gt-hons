A graph $H$ with size at least one is bipartite iff $\chi(G) = 2$ 
and a graph is bipartite iff $H$ contains no odd cycles. 
So if $G$ is minimally 3-chromatic it contains an odd cycle.

Let $v_1, v_2, \ldots, v_{2n+1}$ be the vertices of the smallest odd
cycle in $G$ with $v_iv_{i+1} \in E(G)$. 

Assume that $G$
contains another vertex $u \ne v_i \forall i \in \{1, \ldots, 2n+1\}$
and without loss of generality $uv_1 \in E(G)$. Since $G$ is
critically k-chromatic $\chi(G-u) = 2$. But $\chi(\{ v_1, \ldots, v_{2n+1} \}) = 3$
and $\chi(G-u) = 3 \ne 2$ contradiction. So $G$ cannot contain any
vertices apart from those of the cycle.

Assume that there exist integers $1 \le i < j < 2n+1$ such
that $v_iv_j \in E(G)$. If the difference between $i$ and $j$
is even then $v_1 - \ldots - v_i - v_j - \ldots - v_{2n+1}$
is a smaller odd cycle. Otherwise 
$v_i - v_j - v_{j-1} - \ldots -  v_i$ is a shorter odd cycle.
Contradicting our original choice so our cycle cannot contain
additional vertices apart from those of the cycle. 

Since $G$ cannot contain additional vertices or edges apart from the 
odd cycle$G$ is exactly an odd cycle.

If $G$ is minimally k-chromatic without isolated vertices, by Theorem
42 it is critically k-chromatic and the above applies
