\begin{enumerate}[(a)]
    \item Let $G$ be an outerplanar graph. We then add a vertex $v$ 
    that is adjacent to all vertices in $G$. 
    
    We can see this graph $G+v$ is planar by drawing $v$
    above the embedding of $G$ and joining $v$ with the vertices in $G$ 
    along the outer boundary. These edges don't cross any boundaries of 
    $G$ so $G+v$ is planar.  $G+v \cong G+K_1$.

    Let $G$ not be outerplanar and let $G+K_1$ be planar. We can embed
    $G+K_1$ so that $K_1$ is on the outer boundary of $G+K_1$.
    If we remove $K_1$ from this embedding there should exist a
    vertex $a$ on the interior of $G$ since $G$ is not outerplanar.
    But then the edge $aK_1$ crosses an edge on the outer boundary
    of $G$ making $G+K_1$ not planar, contradiction. Therefore
    $G$ is outerplanar.

    \item Let $G$ be an outerplanar graph of order $n \ge 2$ and size $m$, 
    then $G+K_1$ is planar with size $m+n$ and order $n+1$. Then
    by Corollary 33 
        $$ m+n \le 3(n+1) - 6 = 3n+3-6 = 3n-3$$
        $$m \le 2n-3$$
        as required.

    \item Let $G$ be an outerplanar graph with a subdivision of $K_4$ then by
    6 a) above $G+v$ is planar. But $G+v$ has $K_5$ as a subdivision 
    and by Kuratowski's theorem is not planar. A contradiction. An
    outerplanar graph therefore cannot have a subdivision of $K_4$. 

    The same argument proves that a outerplanar graph can have no
    subdivision of $K_{2,3}$.

    Let $G+K_1$ contain a subdivision of $K_5$ and $x$ be a vertex
    with degree $|V(G)|$. Removing any vertex
    from $K_5$ reduces it to $K_4$. So if we remove $x$ from $G+K_1$
    then $G+K_1-x$ contains $K_4$ but not $K_5$. $G+K_1-x \cong G$.

    The same argument applies to $K_{3,3}$.
\end{enumerate}
